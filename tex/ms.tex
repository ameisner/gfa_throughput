\documentclass{article}

\usepackage{hyperref}
\usepackage{graphicx}
\usepackage{amsmath}
\usepackage{amssymb}

\textwidth=7.0in
\oddsidemargin=-0.25in

\voffset=-0.75in

\textheight=9.0in

\begin{document}

\begin{center}
\Large
DESI GFA Throughput Validation
\end{center}

\normalsize

\begin{center}
Aaron Meisner
\end{center}

\section{GFA Throughput Prediction}

% mention the factors that play in here

In order to validate the system throughput as measured by the DESI GFA cameras, we must first perform an analysis to determine the expected throughput and photometric zeropoint. The primary goal of this analysis is to predict the $r$ band AB magnitude of a source that would correspond to a total detected DESI GFA signal of 1 electron per second. The factors that contribute to determining the DESI GFA throughput are: the atmosphere, the Mayall telescope primary mirror reflectivity, the corrector throughput (including vignetting at the radius of the GFA cameras), the GFA $r$ band filter transmission profile and the GFA detector quantum efficiency (QE). These individual factors are overplotted together in Figure \ref{fig:throughput_factors}.

% calculating the fractional throughput curve: atmosphere, mirror reflectivity, corrector throughput / vignetting, GFA filter transmission, GFA CCD QE


\begin{figure*}[ht]
        \includegraphics[width=7.0in]{gfa_throughput_factors}
    \caption{Summary of the multiplicative factors that contribute to determining the total DESI GFA throughput as a function of wavelength.}
    \label{fig:throughput_factors}
\end{figure*}

The atmospheric transmission is taken from a file called \verb|ZenithExtinction-KPNO.fits|, found in the desimodel product\footnote{\url{https://desi.lbl.gov/svn/code/desimodel/trunk/data/inputs/throughput/ZenithExtinction-KPNO.fits}}. The Mayall telescope primary mirror reflectivity and corrector throughput were taken from ``DESI-347-v14 Throughput Noise SNR Calcs''. The GFA filter transmission was extracted from ``DESI-1297 GFA Filter verification data''. The GFA CCD QE was extracted from the e2v data sheet ``CCD230-42 Back Illuminated Scientific CCD Sensor'', specifically the plot on page 4. Note that the DESI GFA detectors have the `Basic midband coating' option (i.e., the green line in the data sheet's QE plot). The resulting total throughput is shown in Figure \ref{fig:total_throughput}.

\begin{figure*}[ht]
        \includegraphics[width=7.0in]{gfa_total_throughput}
    \caption{Total GFA throughput.}
    \label{fig:total_throughput}
\end{figure*}

% try to figure out where the vignetting value came from...

In order to use the total throughput curve to calculate a predicted zeropoint in AB magnitudes, we additionally need to take into account the mirror area. We adopt a value of 8.66 square meters for the Mayall primary mirror area based on DESI-347-v15. Combining the throughput curve, the definition of the AB magnitude system, and the Mayall primary mirror area we calculate a predicted GFA zeropoint of $r$ = 27.00 AB for a source with total detected signal of 1 electron per second (assuming airmass = 1). This computation is run using the following command:

\verb|calc_ci_zeropoint, /gfa|

\noindent
Where the \verb|calc_ci_zeropoint.pro| code can be found in my \verb|ci_throughput| GitHub repository\footnote{\url{https://github.com/ameisner/ci_throughput}}.

% could add a table listing file name used for each component of the throughput

\section{GFA Gain}

Before we can compare on-sky GFA measurements to our predicted zeropoint, we need to have accurate knowledge of how to convert from ADU (units of the raw GFA data) to electrons. Lab-measured GFA gains are available in a series of DESI DocDB entries, and a set of Mayall dome screen data were also taken specifically to enable post-installation gain measurements on 20191027. My gain measurements from the post-installation dome screen data are documented on the DESI wiki\footnote{\url{https://desi.lbl.gov/trac/wiki/Commissioning/Planning/gfachar/gain_20191027}}. The lab measurements and my dome screen measurements are listed in Table \ref{tab:gains}.

\begin{center}
\begin{table}[ht]
\begin{tabular}{c | c | c | c | c | c | c}
EXTNAME & GFA device & amp & 20191027 gain (e-/ADU) & �acceptance gain (e-/ADU) & acceptance ref. &  region \\ \hline
GUIDE0 & 10 & E & 3.516 & 3.516 & DESI-4750-v1 & all \\
GUIDE0 & 10 & F & 3.575 & 3.545 & DESI-4750-v1 & all \\
GUIDE0 & 10 & G & 3.714 & 3.673 & DESI-4750-v1 & all \\
GUIDE0 & 10 & H & 3.581 & 3.574 & DESI-4750-v1 & all \\
FOCUS1 & 5 & E & 3.589 & 3.559 & DESI-4675-v1 & outer \\
FOCUS1 & 5 & F & 3.790 & 3.712 & DESI-4675-v1 & outer \\
FOCUS1 & 5 & G & 3.686 & 3.644 & DESI-4675-v1 & outer \\
FOCUS1 & 5 & H & 3.867 & 3.795 & DESI-4675-v1 & outer \\
GUIDE2 & 6 & E & 3.686 & 3.668 & DESI-4680-v1 & all \\
GUIDE2 & 6 & F & 3.763 & 3.708 & DESI-4680-v1 & all \\
GUIDE2 & 6 & G & 3.716 & 3.673 & DESI-4680-v1 & all \\
GUIDE2 & 6 & H & 3.788 & 3.733 & DESI-4680-v1 & all \\
GUIDE3 & 2 & E & 3.687 & 3.669 & DESI-4665-v1 & all \\
GUIDE3 & 2 & F & 3.624 & 3.560 & DESI-4665-v1 & all \\
GUIDE3 & 2 & G & 3.852 & 3.785 & DESI-4665-v1 & all \\
GUIDE3 & 2 & H & 3.785 & 3.729 & DESI-4665-v1 & all \\
FOCUS4 & 7 & E & 3.766 & 3.798 & DESI-4713-v1 & outer \\
FOCUS4 & 7 & F & 3.824 & 3.761 & DESI-4713-v1 & outer \\
FOCUS4 & 7 & G & 3.765 & 3.731 & DESI-4713-v1 & outer \\
FOCUS4 & 7 & H & 3.817 & 3.747 & DESI-4713-v1 & outer \\
GUIDE5 & 8 & E & 3.715 & 3.718 & DESI-4716-v1 & all \\
GUIDE5 & 8 & F & 3.792 & 3.741 & DESI-4716-v1 & all \\
GUIDE5 & 8 & G & 3.761 & 3.733 & DESI-4716-v1 & all \\
GUIDE5 & 8 & H & 3.779 & 3.787 & DESI-4716-v1 & all \\
FOCUS6 & 13 & E & 3.744 & 3.740 & DESI-4908-v1 & outer \\
FOCUS6 & 13 & F & 3.715 & 3.670 & DESI-4908-v1 & outer \\
FOCUS6 & 13 & G & 3.720 & 3.698 & DESI-4908-v1 & outer \\
FOCUS6 & 13 & H & 3.749 & 3.732 & DESI-4908-v1 & outer \\
GUIDE7 & 1 & E & 3.814 & 3.863 & DESI-4662-v1 & all \\
GUIDE7 & 1 & F & 3.606 & 3.629 & DESI-4662-v1 & all \\
GUIDE7 & 1 & G & 3.987 & 4.043 & DESI-4662-v1 & all \\
GUIDE7 & 1 & H & 3.906 & 3.967 & DESI-4662-v1 & all \\
GUIDE8 & 4 & E & 3.792 & 3.740 & DESI-4672-v1 & all \\
GUIDE8 & 4 & F & 3.855 & 3.793 & DESI-4672-v1 & all \\
GUIDE8 & 4 & G & 3.610 & 3.601 & DESI-4672-v1 & all \\
GUIDE8 & 4 & H & 3.728 & 3.698 & DESI-4672-v1 & all \\
FOCUS9 & 3 & E & 3.641 & 3.586 & DESI-4747-v2 & outer \\
FOCUS9 & 3 & F & 3.659 & 3.612 & DESI-4747-v2 & outer \\
FOCUS9 & 3 & G & 3.823 & 3.763 & DESI-4747-v2 & outer \\
FOCUS9 & 3 & H & 3.639 & 3.659 & DESI-4747-v2 & outer \\
\end{tabular}
\caption{Comparison of gain measurements from pre-installation lab data taken as part of the GFA acceptance process (R. Casas) and post-installation dome screen data taken on 20191027.}
\label{tab:gains}
\end{table}
\end{center}

% could take out GUIDE cameras from this table...

My gain measurements based on the 20191027 dome screen data are $\sim$0.9\% higher than the corresponding lab gain measurements in the median. David Kirkby has also analyzed the same 20191027 dome screen data and was able to obtain even better agreement with the lab gain measurements (DESI-5315). We therefore adopt the lab gain measurements compiled from the various GFA acceptance reports when comparing out predicted zeropoint to the on-sky measurements. It seems clear that we should not expect our knowledge of the gains to be a significant limitation in the comparison between measured and predicted zeropoints.

\section{Input GFA On-sky Data}

% justification for why this is a good data set to use
%     very close to airmass = 1
%     the unique exposure times in this data set are 30-60 seconds

% make clear that the analysis described here only pertains to the six in-focus guide cameras, and ignores the four out-of-focus cameras
% perhaps in the future i can update the analysis to include checking the zeropoints of the focus cameras as well

\section{Source Catalogs}

\section{PSF-based aperture correction}

% maybe show an example PSF image?

\section{Pan-STARRS Comparison}

The sky location of the frames used is outside of the DESI pre-imaging footprint, but does have Pan-STARRS $r$ band data available.

\section{Zeropoint Values Obtained}

% first appendix
% confirming GFA exptime with star trails

% second appendix all of the per-camera per-amp scatter plots of GFA sources

\begin{center}
\begin{table}[ht]
\begin{tabular}{c | c | c | c | c}
camera & amp & ZP (total raw ADU/s) & ZP (total e-/s) & assumed gain (e-/ADU, acceptance report) \\ \hline
GUIDE0 & E & 25.714 & 27.079 & 3.516 \\
GUIDE0 & F & 25.689 & 27.063 & 3.545 \\
GUIDE0 & G & 25.654 & 27.067 & 3.673 \\
GUIDE0 & H & 25.695 & 27.078 & 3.574 \\
GUIDE2 & E & 25.633 & 27.045 & 3.668 \\
GUIDE2 & F & 25.612 & 27.035 & 3.708 \\
GUIDE2 & G & 25.654 & 27.067 & 3.673 \\
GUIDE2 & H & 25.627 & 27.057 & 3.733 \\
GUIDE3 & E & 25.652 & 27.063 & 3.669 \\
GUIDE3 & F & 25.679 & 27.058 & 3.560 \\
GUIDE3 & G & 25.615 & 27.061 & 3.785 \\
GUIDE3 & H & 25.623 & 27.052 & 3.729 \\
GUIDE5 & E & 25.626 & 27.052 & 3.718 \\
GUIDE5 & F & 25.630 & 27.062 & 3.741 \\
GUIDE5 & G & 25.636 & 27.066 & 3.733 \\
GUIDE5 & H & 25.621 & 27.066 & 3.787 \\
GUIDE7 & E & 25.569 & 27.036 & 3.863 \\
GUIDE7 & F & 25.664 & 27.063 & 3.629 \\
GUIDE7 & G & 25.494 & 27.011 & 4.043 \\
GUIDE7 & H & 25.512 & 27.008 & 3.967 \\
GUIDE8 & E & 25.615 & 27.047 & 3.740 \\
GUIDE8 & F & 25.601 & 27.048 & 3.793 \\
GUIDE8 & G & 25.699 & 27.090 & 3.601 \\
GUIDE8 & H & 25.652 & 27.072 & 3.698 \\
\end{tabular}
\caption{zeropoint results}
\label{tab:zps}
\end{table}
\end{center}


\appendix

\section{Confirming GFA EXPTIME}

\section{Per-camera, per-amp scatter plots}

\end{document}